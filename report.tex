\documentclass[a4paper,11pt]{article}

\usepackage{cite}
\usepackage[T1]{fontenc}
\usepackage[utf8]{inputenc}
\usepackage{graphicx}
\usepackage{fancyheadings}
\usepackage{geometry}
\usepackage{multicol}
\usepackage{amsmath}
%\usepackage[]{mcode}
\usepackage{pdfpages}
\usepackage{caption} 
\captionsetup[table]{skip=10pt}

 \newcommand{\ba}[1]{\begin{align*}    #1    \end{align*}}
 \newcommand{\ban}[1]{\begin{align}    #1    \end{align}}
 \renewcommand{\vec}[1]{\mathbf{#1}}
 \newcommand{\bothset}[3]{\overset{#2}{\underset{#3}{#1}}}

 \definecolor{light-gray}{gray}{0.95}

\setlength{\headsep}{1in}

\pagestyle{fancy}

\lhead{ridenour@kth.se \\ herczka@kth.se}
\rhead{Parallel Programming for Large-Scale Problems SF2568}
\title{ An Explicit Parallel Solver for the Gray-Scott Reaction-Diffusion System\\ 
\vspace{1.2cm}
\large Parallel Programming for Large-Scale Problems SF2568 \\ 
Teacher: Michael Hanke}
\author{Jonathan Ridenour, 780514-7779\\
Mateusz Herczka, 700624-9234}

\begin{document}
%\lstdefinestyle{customc}{
% backgroundcolor=\color{light-gray},
%  belowcaptionskip=1\baselineskip,
%  breaklines=true,
%  frame=L,
%  xleftmargin=\parindent,
%  language=C,
%  showstringspaces=false,
%  basicstyle=\footnotesize\ttfamily,
%  keywordstyle=\bfseries\color{green!40!black},
%  commentstyle=\itshape\color{purple!40!black},
%  identifierstyle=\color{blue},
%  stringstyle=\color{orange},
%}
%
%\lstdefinestyle{output}{
% backgroundcolor=\color{light-gray},
% language=bash,
% xleftmargin=\parindent,
% keywordstyle=\color{blue},
% basicstyle=\ttfamily,
% morekeywords={peter@kbpet},
% %alsoletter={:~$},
% morekeywords=[2]{peter@kbpet:},
%}
\maketitle
\pagebreak
\section*{Problem description}
The subject of this report is the numerical solution of a two-component reaction-diffusion system known as the Gray-Scott system \cite{Wang}. 
\subsection*{mathematical formulation}
\subsection*{numerical method}
We proceed by means of a mixed implicit-explicit scheme combining the Crank-Nicolson and Adams-Bashforth methods, as is considered in \cite{Ruuth}.  The fully discretized system is written
\ba{
\frac{u_{i,j}^{n+1}-u_{i,j}^{n+1}}{dt} = & \ \frac{Du}{h^2}\bigg( u_{i-1,j}^{n+1} + u_{i+1,j}^{n+1} + u_{i,j-1}^{n+1}+ u_{i,j+1}^{n+1} - 4u_{i,j}^{n+1} \bigg) \\
+ & \ \frac{Du}{h^2}\bigg( u_{i-1,j}^{n} + u_{i+1,j}^{n} + u_{i,j-1}^{n}+ u_{i,j+1}^{n} - 4u_{i,j}^{n} \bigg) \\
+ & \ \frac{3}{2} \bigg( -u_{i,j}^{n} (v_{i,j}^{n})^2 + f(1-u_{i,j}^{n}) \bigg) \\
- & \ \frac{1}{2} \bigg( -u_{i,j}^{n-1} (v_{i,j}^{n-1})^2 + f(1-u_{i,j}^{n-1}) \bigg), \\ \\
\frac{v_{i,j}^{n+1}-v_{i,j}^{n+1}}{dt} = & \ \frac{Dv}{h^2}\bigg( v_{i-1,j}^{n+1} + v_{i+1,j}^{n+1} + v_{i,j-1}^{n+1}+ v_{i,j+1}^{n+1} - 4v_{i,j}^{n+1} \bigg) \\
+ & \ \frac{Dv}{h^2}\bigg( v_{i-1,j}^{n} + v_{i+1,j}^{n} + v_{i,j-1}^{n}+ v_{i,j+1}^{n} - 4v_{i,j}^{n} \bigg) \\
+ & \ \frac{3}{2} \bigg( -u_{i,j}^{n} (v_{i,j}^{n})^2 - (f+k)v_{i,j}^{n} \bigg) \\
- & \ \frac{1}{2} \bigg( -u_{i,j}^{n-1} (v_{i,j}^{n-1})^2  - (f+k)v_{i,j}^{n-1} \bigg).
}
If we let $\sigma_u = D_u dt/h^2$ and $\sigma_v = D_v dt/h^2$ these equations can be rewritten
\ba{
-\sigma_u u_{i+1,j}^{n+1} - & \ \sigma_u u_{i-1,j}^{n+1} -\sigma_u u_{i,j+1}^{n+1} -\sigma_u u_{i,j-1}^{n+1} + (1+ 4 \sigma) u_{i,j}^{n+1}\\ 
= & \ \sigma_u u_{i+1,j}^{n}+\sigma_u u_{i-1,j}^{n}+\sigma_u u_{i,j+1}^{n}+\sigma_u u_{i,j-1}^{n} + (1- 4 \sigma) u_{i,j}^{n+1}\\
+ & \ F,
}
\section*{Psuedo-code algorithm description}
\section*{Theoretical performance evaluation}
\section*{Implementation details}
\section*{Testing with different cases}
\section*{Experimental speedup investigation with meaningful data}

\begin{thebibliography}{}
\bibitem{Wang} Weiming Wang, Yezhi Lin, Feng Yang, Lei Zhang, Yongji Tan, {\it Numerical study of pattern formation in an extended Gray–Scott model}, Communications in Nonlinear Science and Numerical Simulation, 16 (2011).
\bibitem{Ruuth} Steven J Ruuth, {\it Implicit-Explicit Methods for Reaction-Diffusion Problems in Pattern Formation}, Journal of Mathematical Biology 34 (2), 148-176. 
%\bibitem{Wilkinson} Barry Wilkinson, Michael Allen {\it Parallel Programming, Techniques and Applications Using Networked Workstations and Parallel Computers}, 2005, Pearson Prentice Hall, pp 9.
\end{thebibliography}

\end{document}