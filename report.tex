\documentclass[a4paper,11pt]{article}

\usepackage{cite}
\usepackage[T1]{fontenc}
\usepackage[utf8]{inputenc}
\usepackage{graphicx}
\usepackage{fancyheadings}
\usepackage{geometry}
\usepackage{multicol}
\usepackage{amsmath}
\usepackage{amssymb}
\usepackage{pdfpages}
\usepackage{caption} 
\usepackage[]{algorithm2e}
\captionsetup[table]{skip=10pt}

 \newcommand{\ba}[1]{\begin{align*}    #1    \end{align*}}
 \newcommand{\ban}[1]{\begin{align}    #1    \end{align}}
 \renewcommand{\vec}[1]{\mathbf{#1}}
 \newcommand{\bothset}[3]{\overset{#2}{\underset{#3}{#1}}}

 \definecolor{light-gray}{gray}{0.95}

\setlength{\headsep}{1in}

\pagestyle{fancy}

\lhead{ridenour@kth.se \\ herczka@kth.se}
\rhead{Parallel Programming for Large-Scale Problems SF2568}
\title{ Analysis of a Parallel Solver for the Gray-Scott Reaction-Diffusion System\\ 
\vspace{1.2cm}
\large Parallel Programming for Large-Scale Problems SF2568 \\ 
Teacher: Michael Hanke}
\author{Jonathan Ridenour, 780514-7779\\
Mateusz Herczka, 700624-9234}

\begin{document}
\maketitle
\pagebreak
\section*{Problem description}
The subject of this report is the parallelization of a numerical method for a two-component reaction-diffusion system known as the Gray-Scott model.  Originally introduced by Gray and Scott \cite{Gray}, the model is a system of two partial differential equations for the concentration of two chemical species, which are reacting with each other as they diffuse through a medium.

Pattern formation resulting from the Gray-Scott model is a rich area of research, with applications to a variety of physical, chemical, and biological phenomena, as well as cellular automata and the pattern formation of other partial differential equation systems \cite{Wang}.  We investigate a parallelization strategy for the two-dimensional case.  We make a series of trials, running the solver in parallel on 4, 16, 64, and 100 processes, and perform a speedup analysis using the results of these trials. Finally, we show a few of the various patterns that emerge from the Gray-Scott model with various input parameters.

\subsection*{Mathematical Formulation}
The Gray-Scott model involves the reaction and diffusion of two generic chemical species, $U$ and $V$, whose concentrations are described by the functions $u$ and $v$, reacting according to the chemical equations
\ba{
U + 2V \rightarrow  & \ 3V,\\
V \rightarrow  & \ P,
}
where $P$ is an inert byproduct.  This system is governed by the following system of partial differential equations, known as the Gray-Scott equations:
\begingroup
\addtolength{\jot}{0.5 em}
\ban{
\label{eq:gsu}
\frac{\partial u}{\partial t} = & \ D_u \nabla^2u - uv^2 + F(1-u), & \ u,v: \Omega \mapsto \mathbb{R}, \ t\ge 0,  \\
\label{eq:gsv}
\frac{\partial v}{\partial t} = & \ D_v \nabla^2v + uv^2 - (F-K)v , & \ u,v: \Omega \mapsto \mathbb{R}, t \ \ge 0.
}
\endgroup
Here, $D_u$ and $D_v$ are the diffusion constants for $u$ and $v$ respectively, and $F$ and $K$ are constants which govern the replenishment of the chemical species.  The term $uv^2$ gives the reaction rate for the system.  We are concerned with solving the Gray-Scott equations on a two-dimensional domain $\Omega = [0,1]^2$.  Thus, $u = u(x,y,t)$ and $v = v(x,y,t)$, giving the laplacian as
\ba{
\nabla^2u = & \ \frac{\partial^2 u}{\partial x^2} + \frac{\partial^2 u}{\partial y^2}, \\
\nabla^2u = & \ \frac{\partial^2 u}{\partial x^2} + \frac{\partial^2 u}{\partial y^2}.
}

As in \cite{Wang}, we use periodic boundary conditions for (\ref{eq:gsu}) and (\ref{eq:gsv}); thus, our simulation on $\Omega$ models a small surface surrounded by similar elements reacting identically. We choose the diffusion constants $D_u = 2 \cdot 10^{-6}$, $D_v = 1 \cdot 10^{-6}$, and let $F$ and $K$ take values according to $F \in [0.02, \ 0.08]$, $K \in [0.05,\ 0.07]$ in order to investigate the different types of patterns that arise from varying combinations of the two.

The system is in a trivial state when $u = 1$ and $v=0$.  To initialize a reaction, we let the system assume this trivial state at time zero in all areas except a square in the middle of dimension one-sixth, where we set $u = 0.5$ and $v = 0.25$:
\ban{
\label{eq:icu}
u(x,y,0) = \begin{cases}
1/2, & \ 5/12 \ge x \ge 7/12, \ 5/12 \ge y \ge 7/12,\\
1, & \ \text{otherwise},
 \end{cases}\\
 \label{eq:icv}
 v(x,y,0) = \begin{cases}
1/4, & \ 5/12 \ge x \ge 7/12, \ 5/12 \ge y \ge 7/12,\\
0, & \ \text{otherwise}.
 \end{cases}
}
The reaction then diffuses outward from the middle, leaving behind a characteristic pattern, until all $\Omega$ is filled.

\subsection*{Numerical Method}
We discretize $\Omega$ with the same stepsize in the $x$- and $y$-directions, $h$, corresponding to $N^2$ inner gridpoints equidistantly spread across the surface.  We let $h = 1/(N+1)$ such that $x_i = ih$, $i$ = 1, 2, . . . , $N$ and $y_j = jh$, $j$ = 1, 2, . . . , $N$.  Thus the function value $u(x,y,t)$ is approximated by $u(ih, jh, t) = u_{i,j}(t)$, $t\ge 0$. Using the traditional central difference formula for the discrete five-point laplacian with step length $h$, $\Delta_5^h$, the system (\ref{eq:gsu}) + (\ref{eq:gsv}) can be written in semi-discrete form as follows: 
\begingroup
\addtolength{\jot}{0.5 em}
\ban{
\label{eq:semiu}
\frac{\partial u_{i,j}}{\partial t} = \frac{D_u}{h^2} \Delta_5^h u_{i,j} -u_{i,j} (v_{i,j})^2 + F(1-u_{i,j}), \\
\label{eq:semiv}
\frac{\partial v_{i,j}}{\partial t} = \frac{D_v}{h^2} \Delta_5^h v_{i,j} +u_{i,j} (v_{i,j})^2 - (F-K)v_{i,j}.
}
\endgroup
We then discretize in time, with timestep $dt$, and approximate the time-derivative with the explicit Euler method.  Letting $t = k\cdot dt$ for $k$ = 1, 2, 3, . . . ., and $u_{i,j}(k\cdot dt) = u_{i,j}^k$, $v_{i,j}(k\cdot dt) = v_{i,j}^k$, the system (\ref{eq:semiu}) + (\ref{eq:semiv})  becomes
\begingroup
\addtolength{\jot}{0.5 em}
\ban{
\label{eq:difu}
\frac{u_{i,j}^{k+1} - u_{i,j}^k}{dt} = \frac{D_u}{h^2} \Delta_5^h u_{i,j}^k -u_{i,j}^k (v_{i,j}^k)^2 + F(1-u_{i,j}^k), \\
\label{eq:difv}
\frac{v_{i,j}^{k+1} - v_{i,j}^k}{dt} = \frac{D_v}{h^2} \Delta_5^h v_{i,j}^k +u_{i,j}^k (v_{i,j}^k)^2 - (F-K)v_{i,j}^k.
}
\endgroup
The difference equations (\ref{eq:difu}) and (\ref{eq:difv}) give a simple formula for the update of the function values at time $(k+1) \cdot dt$ given that values at time $k \cdot dt$.  We represent the discrete functions $u_{i,j}^k$ and  $v_{i,j}^k$ as long vectors $\vec{u}^k$ and $\vec{v}^k$ with the following enumeration:
\ba{
\vec{u}^k = & \ (u_{1,1}^k, \ u_{2,1}^k, \ \text{. . . }, \ u_{N,1}^k, \ u_{1,2}^k, \ u_{2,2}^k, \ \text{. . . }, \ u_{N,N}^k), \\
\vec{v}^k = & \ (v_{1,1}^k, \ v_{2,1}^k, \ \text{. . . }, \ v_{N,1}^k, \ v_{1,2}^k, \ v_{2,2}^k, \ \text{. . . }, \ v_{N,N}^k).
}
Using the long vectors, we can write the numerical operator $\Delta_5^h$ as a sparse matrix multiplication as follows:
\begingroup
\addtolength{\jot}{0.5 em}
\ba{
\frac{D_u}{h^2} \Delta_5^h u_{i,j} = \vec{A}_u\vec{u}^n,\\
\frac{D_v}{h^2} \Delta_5^h v_{i,j} = \vec{A}_v\vec{v}^n,
}
\endgroup
where the matrices $\vec{A}_u$ and $\vec{A}_v$ are of dimension $[N^2 \times N^2]$ with block tridiagonal form:
\ba{
\vec{A}_u = & \ \text{tridiag}_N(\sigma_u\vec{I}, \ \vec{T}_u, \ \sigma_u\vec{I}), \\
\vec{A}_v = & \ \text{tridiag}_N(\sigma_v\vec{I}, \ \vec{T}_v, \ \sigma_v\vec{I}),
}
where $\vec{I}$ is an identity matrix, multiplied by the constants $\sigma_u$ or $\sigma_v$, and the matrices $\vec{T}_u$ and $\vec{T}_v$ are tridiagonal of the form
\ba{
\vec{T}_u = & \ \text{tridiag}_N(\sigma_u, \ -4\sigma_u, \ \sigma_u), \\
\vec{T}_v = & \ \text{tridiag}_N(\sigma_v, \ -4\sigma_v, \ \sigma_v),
}
where 
\begingroup
\addtolength{\jot}{0.5em}
\ba{
\sigma_u =& \ \frac{D_u}{h^2}, \\
\sigma_v =& \ \frac{D_v}{h^2}.
}
\endgroup
If we let $f_u(\vec{u}^{k},\vec{v}^{k}) = -\vec{u}^k (\vec{v}^k)^2 + F(1-\vec{u}^k)$ and $ f_v(\vec{u}^{k},\vec{v}^{k}) = \vec{u}^k (\vec{v}^k)^2 - (F-k)\vec{v}^k$, we can compactly write the fully discretized equations as
\ban{
\label{eq:discu}
\vec{u}^{k+1} = \vec{u}^{k} + dt [\vec{A}_u \vec{u}^{k} + f_u(\vec{u}^{k},\vec{v}^{k})],\\
\label{eq:discv}
\vec{v}^{k+1} = \vec{v}^{k} + dt [\vec{A}_v \vec{v}^{k} + f_v(\vec{u}^{k},\vec{v}^{k})],
}
and we have an exact update formula for each time step.
\subsubsection*{Initial and Boundary Conditions}
The initial and boundary conditions are straightforward to implement numerically.  The initial state is completely known, so $\vec{u}^0$ and $\vec{v}^0$ are given.  To implement periodic boundary conditions we simply set
\ba{
u_{-1,j}^k = & \ u_{N-1,j}^k, \\
u_{i,-1}^k =& \  u_{i,N-1}^k, \\
u_{N,j}^k =& \  u_{0,j}^k,\\ 
u_{i,N}^k = & \ u_{i,0}^k,\\ 
}
when updating the boundary points.  The formula is likewise for $v$.

\subsubsection*{Stability Conditions}
In order to ensure numerical stability for the differential operators $\vec{A}_u$ and $\vec{A}_v$ in  (\ref{eq:discu}) and (\ref{eq:discv}), we must conform to the restriction
\ba{
dt \cdot \lambda_{l,u} \in & \ \mathcal{S}, \ \forall  \ l, \ l = \text{1, 2, . . . $L_u$},\\
dt \cdot \lambda_{l,v} \in & \ \mathcal{S}, \ \forall  \ l, \ l = \text{1, 2, . . . $L_v$},
}
where $\lambda_{l,u}$ and $\lambda_{l,v}$ are the $l$-th eigenvalues of $\vec{A}_u$ and $\vec{A}_v$ respectively,  and $\mathcal{S}$ is the stability region of the explicit Euler method: a circle in the complex plane centered at $-1$ with unit radius \cite{Edsberg}.  The number of unique eigenvalues possessed by each of the matrices is denoted by $L_u$ and $L_v$.   

For the parallel speedup trials, we discretize $\Omega$ using $N = 400$.  Together with the chosen diffusion constants, the  maximum eigenvalue, $\lambda_{max}$, of $\vec{A}_u$ and $\vec{A}_v$ has a strictly negative real value, and thus we must choose a timestep such that $dt \cdot \lambda_{max}$ is at least $-2$.  The maximum eigenvalue is computed using Matlab as:
\ba{
\lambda_{max} = -1.2735,
}
which provides the limitation on the timestep:
\ba{
dt \le \frac{-2}{-1.2735} = 1.5704.
}
Since the timestep also has a damping effect on the replenishment term and reaction rate, we must consider these as well when choosing the timestep.  As in \cite{Wang}, we settle on a value which is comfortably beneath the max:
\ba{
dt = 0.75.
}
This is well within the stability bounds imposed by $\vec{A}_u$ and $\vec{A}_v$; it also provides suitable damping on the replenishment and reaction rate terms for the observation of the characteristic Gray-Scott pattern formation.

\section*{Algorithm Description}
\subsection*{Implementation details}
To implement the update formulae (\ref{eq:discu}) and (\ref{eq:discv}) in parallel, the computational domain is divided into subdomains, one for each process.  Each process is responsible for the update of it's local values.  Along the subdomain boundaries, neighbouring values are needed for the update; these are obtained by coordinated message-passing.  The boundary data are communicated between processes by means of a red-black communication schedule.  When all necessary information is obtained, the local values are updated according to (\ref{eq:discu}) and (\ref{eq:discv}), see Algorithm \ref{alg:coms}.

For a domain size of $400 \times 400$, the global update boils down to two matrix multiplications, one using $\vec{A}_u$, one using $\vec{A}_v$, each of which contains $400^4$ elements, most of which are zero.  This can be accomplished by means of a sparse matrix multiplication method, or .... fill in your part here!

\vspace{0.7 cm}
\begin{algorithm}[H]
\label{alg:coms}
 \KwData{$dt$, $\vec{u}^k$, $\vec{v}^k$, $\vec{A}_u$, $\vec{A}_v$, red}
 \KwResult{$\vec{u}^{k+1}$ and $\vec{v}^{k+1}$ }
collect boundary data $\vec{b}_{out}$\;
  \eIf{red}{
send $\vec{b}_{out}$: north, south, east, west\;
receive $\vec{b}_{in}$: south, north, west, east\;
   }{
receive $\vec{b}_{in}$: south, north, west, east\;
send $\vec{b}_{out}$: north, south, east, west\;
  }
$\vec{u}^{k+1} = \vec{u}^k+ dt [\vec{A}_u \vec{u}^k +  f_u(\vec{u}^{k},\vec{v}^{k})] + \vec{b}_{in}$\;
$\vec{v}^{k+1} = \vec{v}^k+ dt [\vec{A}_v \vec{v}^k +  f_v(\vec{u}^{k},\vec{v}^{k})] + \vec{b}_{in}$\;
 \vspace{0.3 cm}
\caption{Communication and local update.}
\end{algorithm}
\vspace{0.7 cm}

\section*{Performance evaluation}
\subsection*{Extrapolated speedup}
\subsection*{Experimental speedup}
\begin{table}[h]
\def\arraystretch{1.2}
\begin{center}
\caption{Max time (seconds) measurements, kernel method.}
\label{tab:kernal}
\vspace{0.3cm}
\begin{tabular}{| c | c | c | c | c |}
\hline
$P$ & $t_{init}$ & $t_{comm}$ & $t_{calc}$ & $t_{tot}$ \\
\hline
1 & 0.0046 & 0.0000 &113.0 &\\
\hline
4 & 0.0051& 3.157 & 91.37 &\\
\hline
16 & 0.1540 & 3.115 & 6.559 &\\
\hline
64 & 0.1845 & 2.410 &1.683 &\\
\hline
100 & & &  &\\
\hline
\end{tabular}
\end{center}
\end{table}

\section*{Patterns}
\section*{Conclusions}

\begin{table}[h]
\def\arraystretch{1.2}
\begin{center}
\caption{Max time (seconds) measurements, sparse matrix method.}
\label{tab:matrix}
\vspace{0.3cm}
\begin{tabular}{| c | c | c | c | c |}
\hline
$P$ & $t_{init}$ & $t_{comm}$ & $t_{calc}$ & $t_{tot}$ \\
\hline
1 & 0.1437 &  0.000 & 793.0 & 703.14\\
\hline
4 & 0.05663 & 36.48 & 301.8 & 304.4\\
\hline
16 & 0.008907 & 8.786 & 38.42 & 40.87\\
\hline
64 & 0.009324 & 2.8460 & 7.391 & 9.603\\
\hline
100 & 0.001253 & 3.206 & 4.287 & 7.419\\
\hline
\end{tabular}
\end{center}
\end{table}

\clearpage
\begin{thebibliography}{}
\bibitem{Gray} Gray P, Scott SK. {\it Sustained oscillations and other exotic patterns of behavior in isothermal reactions}. J Phys Chem 1985;89:22–32.
\bibitem{Wang} Weiming Wang, Yezhi Lin, Feng Yang, Lei Zhang, Yongji Tan, {\it Numerical study of pattern formation in an extended Gray–Scott model}, Communications in Nonlinear Science and Numerical Simulation, 16 (2011).
%\bibitem{Ruuth} Steven J Ruuth, {\it Implicit-Explicit Methods for Reaction-Diffusion Problems in Pattern Formation}, Journal of Mathematical Biology 34 (2), 148-176. 
\bibitem{Edsberg} Lennart Edsberg {\it Introduction to Computation and Modelling for Differential Equations}, 2008, John Wiley and Sons, pp 50.
%\bibitem{Wilkinson} Barry Wilkinson, Michael Allen {\it Parallel Programming, Techniques and Applications Using Networked Workstations and Parallel Computers}, 2005, Pearson Prentice Hall, pp 9.
\end{thebibliography}

\end{document}